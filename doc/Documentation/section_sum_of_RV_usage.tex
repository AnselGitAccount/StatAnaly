\section{Usage in C++}

A good place to see example usages is the gtest files \textit{tests/tst\_dConvolution.cpp} \\
and \textit{tests/tst\_dConvolution\_squares.cpp}.

There are three global static variables representing different sums:
\begin{enumerate}
    \item \(\it{cnvl}\) represents the simple sum, ie $R=X+Y$.
    \item \(\it{cnvlSq}\) represents the sum of the squares, ie $R=X^2+Y^2$.
    \item \(\it{cnvlSSqrt}\) represents the sum of the squares and then take square root, ie $R=\sqrt{X^2+Y^2}$.
\end{enumerate}


\subsection{Simple Sum}

\subsubsection*{Example 1}
The sum of two RVs of standard uniform distributions is a RV of Irwin-Hall distribution.

\begin{minted}{cpp}
disStdUniform su1{}, su2{};
probDistr* rsu = cnvl.go(su1, su2);
disIrwinHall* rih = dynamic_cast<disIrwinHall*>(rsu);
\end{minted}

\subsubsection*{Example 2}
The sum of an array of RVs of standard uniform distributions is a RV of Irwin-Hall distribution.

\begin{minted}{cpp}
disStdUniform su1{}, su2{}, su3{}, su4{};
probDistr* rsu = convolve<disStdUniform>({su1, su2, su3, su4});
disIrwinHall* rih = dynamic_cast<disIrwinHall*>(rsu);
\end{minted}

%%%%%%%%%%%%%%%%%%%%%%%%%%%%%%%%%%%%%%%%%%%%%%%%%%%%%%

\subsection{Sum of the Squares}

\subsubsection*{Example}
The sum of two squares of RVs of normal distributions is a RV of Chi Square distribution.

\begin{minted}{cpp}
disNormal a(0,1);
disNormal b(0,1);
disChiSq s = *static_cast<disChiSq*>(cnvlSq.go(a,b));
\end{minted}

%%%%%%%%%%%%%%%%%%%%%%%%%%%%%%%%%%%%%%%%%%%%%%%%%%%%%%

\subsection{Sum of the Squares, and Then Take Square Root}

\subsubsection*{Example}
The sum of two squares of RVs of normal distribution (with same variance) is a RV of Rician distribution.

\begin{minted}{cpp}
disNormal a(2,9);
disNormal b(3,9);
disRician s = *static_cast<disRician*>(cnvlSSqrt.go(a,b));
\end{minted}


%%%%%%%%%%%%%%%%%%%%%%%%%%%%%%%%%%%%%%%%%%%%%%%%%%%%%%
%%%%%%%%%%%%%%%%%%%%%%%%%%%%%%%%%%%%%%%%%%%%%%%%%%%%%%

\section{How to add a new sum of random variables?}

If you do not find your distributions in currenctly supported list, you can easily add a new sum for your random variables.

... is implemented in the double dispatcher pattern. (See \textit{Modern C++ Design by Andrei Alexandrescu} for pattern detail.)

TBW



\begin{enumerate} 
    \item Register the pair in \textit{dConvolution.cpp}.
    \item Write a function prototype in \textit{dConvolution.h}.
    \item Implement the sum in \textit{dConvolution.cpp}.
\end{enumerate}

\begin{minted}{cpp}
/* dConvolution.cpp */

auto ConvolutionSqDoubleDispatcherInitialization = [](){
    cnvlSq.add<disNormal,disNormal,convolveSq>();
    cnvlSq.add<disApple,disBananna,convolveSq>();  // new distribution pair.
    return true;
}();

probDistr* convolve(disApple& l, disBananna& r) {
    // Implement the sum
    // Blah, blah, blah 
    probDistr* res = xxxxx;
    return res;
};
\end{minted}


\begin{minted}{cpp}
/* dConvolution.h */

probDistr* convolve(disCauchy& lhs, disCauchy& rhs);
\end{minted}