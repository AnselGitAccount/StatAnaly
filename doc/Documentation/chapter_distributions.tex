\chapter{Probability Distribution}
\label{ch:prob_distr}

\section{List of supported probability distributions}
StatAnaly supports the following probability distributions, in alphabetical order:

\begin{enumerate}
    \item Cauchy distribution
    \item Chi distribution
    \item Chi Squared distribution
    \item Erlang distribution
    \item Exponential distribution
    \item Gamma distribution
    \item Irwin-Hall distribution
    \item Non-central Chi distribution
    \item Non-central Chi Squared distribution
    \item Normal distribution
    \item Rayleigh distribution
    \item Standard Normal distribution
    \item Uniform distribution (Continuous)
\end{enumerate}


%%%%%%%%%%%%%%%%%%%%%%%%%%%%%%%%%%%%%%%%%%%%%%%%%%%%%%%%%%%%%%%%%%
\subsection{Cauchy distribution}

\subsubsection*{PDF}
\[
    P(x) = \frac{1}{\pi} \frac{b}{(x-m)^2 + b^2}
\]
where $b$ is the scale parameter which specifies the half-width at half-maximum; 
$m$ is the location parameter which specifies the location of the peak of the distribution.

\subsubsection*{CDF}
\[
    D(x) = \frac{1}{\pi} \arctan{(\frac{x-m}{b})} + \frac{1}{2}
\]

\subsubsection*{Mean}
\[
    \text{Undefined}
\]

\subsubsection*{Variance}
\[
    \text{Undefined}
\]

\subsubsection*{Skewness}
\[
    \text{Undefined}
\]

Reference:
\href{https://mathworld.wolfram.com/CauchyDistribution.html}{WolframMathWorld}

%%%%%%%%%%%%%%%%%%%%%%%%%%%%%%%%%%%%%%%%%%%%%%%%%%%%%%%%%%%%%%%%%%
\subsection{Chi distribution} 

\subsubsection*{PDF}
\[
    P(x) = \frac{2^{1-k/2} x^{k-1} \exp{(-x^2/2)}}{\Gamma(k/2)}
\]
where $k$ is the degrees of freedom.

\subsubsection*{CDF}
\[
    D(x) = P(k/2, x^2/2)
\]
where $P(a,b)$ is the regularized gamma function.

\subsubsection*{Mean}
\[
    \mu = \sqrt{2} \frac{\Gamma((k+1)/2)}{\Gamma(k/2)}
\]

\subsubsection*{Variance}
\[
    \sigma^2 = k-\mu^2
\]

\subsubsection*{Skewness}
\[
    \frac{\mu}{\sigma^3} (1-2\sigma^2)
\]

Reference:
\href{https://en.wikipedia.org/wiki/Chi_distribution}{Wikipedia}

%%%%%%%%%%%%%%%%%%%%%%%%%%%%%%%%%%%%%%%%%%%%%%%%%%%%%%%%%%%%%%%%%%
\subsection{Chi Squared distribution}

A Chi Squared distribution with $k$ degrees of freedom is the 
distribution of a sum of the squares of $k$ independent 
standard normal random variables.

\subsubsection*{PDF}
\[
    P(x) = \frac{x^{k/2-1} \exp(-x/2)}{2^{k/2} \Gamma(k/2)}
\]
where $k$ is the degrees of freedom; $\Gamma(a)$ is the gamma function.

\subsubsection*{CDF}
\[
    D(x) = P(k/2, x/2)
\]
where $P(a,b)$ is the regularized gamma function.

\subsubsection*{Mean}
\[
    \mu = k
\]

\subsubsection*{Variance}
\[
    2k
\]

\subsubsection*{Skewness}
\[
    \sqrt{8/k}
\]

Reference:
\href{https://en.wikipedia.org/wiki/Chi-squared_distribution}{Wikipedia}

%%%%%%%%%%%%%%%%%%%%%%%%%%%%%%%%%%%%%%%%%%%%%%%%%%%%%%%%%%%%%%%%%%
\subsection{Erlang distribution}

\subsubsection*{PDF}
\[
    P(x) = \frac{\lambda^k x^{k-1} \exp{(\lambda x)}}{(k-1)!}
\]
where $k$ is the shape parameter; $\lambda$ is the rate parameter.

\subsubsection*{CDF}
\[
    D(x) = \frac{\gamma(k, \lambda x)}{(k-1)!}
\]
where $\gamma(a,b)$ is the lower gamma function.

\subsubsection*{Mean}
\[
    \frac{k}{\lambda}
\]

\subsubsection*{Variance}
\[
    \frac{k}{\lambda^2}
\]

\subsubsection*{Skewness}
\[
    \frac{2}{\sqrt{k}}
\]

Reference:
\href{https://en.wikipedia.org/wiki/Erlang_distribution}{Wikipedia}

%%%%%%%%%%%%%%%%%%%%%%%%%%%%%%%%%%%%%%%%%%%%%%%%%%%%%%%%%%%%%%%%%%
\subsection{Exponential distribution}

\subsubsection*{PDF}
\[
    P(x) = \begin{cases} 
        \lambda \exp{(-\lambda x)} & x\geq 0,\\
        0 & x<0.
        \end{cases}
\]

\subsubsection*{CDF}
\[
    D(x) = \begin{cases}
        1-\exp{(-\lambda x)} & x \ge 0, \\
        0 & x < 0.
        \end{cases}
\]

\subsubsection*{Mean}
\[
    \frac{1}{\lambda}
\]

\subsubsection*{Variance}
\[
    \frac{1}{\lambda^2}
\]

\subsubsection*{Skewness}
\[
    2
\]

Reference:
\href{https://en.wikipedia.org/wiki/Exponential_distribution}{Wikipedia}

%%%%%%%%%%%%%%%%%%%%%%%%%%%%%%%%%%%%%%%%%%%%%%%%%%%%%%%%%%%%%%%%%%
\subsection{Gamma distribution}

\subsubsection*{PDF}
\[
    P(x)= \frac{1}{\Gamma(\alpha)\theta^{\alpha}} 
            x^{\alpha-1} \exp{(-\frac{x}{\theta})}
\]
where $\alpha$ is the shape parameter; $\theta$ is the scale parameter.

\subsubsection*{CDF}
\[
    D(x)= \frac{1}{\Gamma(\alpha)} \gamma(\alpha, \frac{x}{\theta})
\]

\subsubsection*{Mean}
\[
    k \theta
\]

\subsubsection*{Variance}
\[
    k \theta^2
\]

\subsubsection*{Skewness}
\[
    \frac{2}{\sqrt(\alpha)}
\]

Reference:
\href{https://mathworld.wolfram.com/GammaDistribution.html}{WolframMathWorld}

%%%%%%%%%%%%%%%%%%%%%%%%%%%%%%%%%%%%%%%%%%%%%%%%%%%%%%%%%%%%%%%%%%
\subsection{Irwin-Hall distribution}

\subsubsection*{PDF}
\[
    P(x) = \frac{1}{(n-1)!} \sum_{{k=0}}^{{\lfloor x\rfloor }}
        (-1)^{k}{\binom{n}{k}}(x-k)^{{n-1}}
\]
where $n$ is number of IDD of uniform distributions.

\subsubsection*{CDF}
\[
    D(x) = \frac{1}{n!} \sum_{{k=0}}^{{\lfloor x\rfloor}}
        (-1)^{k}{\binom{n}{k}}(x-k)^{n}
\]

\subsubsection*{Mean}
\[
    \frac{n}{2}
\]

\subsubsection*{Variance}
\[
    \frac{n}{12}
\]

\subsubsection*{Skewness}
\[
    0
\]

Reference:
\href{https://en.wikipedia.org/wiki/Irwin%E2%80%93Hall_distribution}{WolframMathWorld}

%%%%%%%%%%%%%%%%%%%%%%%%%%%%%%%%%%%%%%%%%%%%%%%%%%%%%%%%%%%%%%%%%%
\subsection{Non-central Chi distribution}

\subsubsection*{PDF}
\[
    P(x) = {\frac{\exp{(-(x^{2}+\lambda^{2})/2)}x^{k}\lambda}
        {(\lambda x)^{k/2}}}
        I_{k/2-1}(\lambda x)
\]
where $k$ is the degrees of freedom; $\lambda$ is the distance parameter;
$I_M(a)$ is a modified cylindrical Bessel function of the first kind.

\subsubsection*{CDF}
\[
    D(x) = 1-Q_{\frac {k}{2}}\left(\lambda ,x\right)
\]
where $Q_M(a,b)$ is Marcum Q-function.

\subsubsection*{Mean}
\[
    \text{To be implemented.}
\]

\subsubsection*{Variance}
\[
    \text{To be implemented.}
\]

Reference:
\href{https://en.wikipedia.org/wiki/Noncentral_chi_distribution}{Wikipedia}

%%%%%%%%%%%%%%%%%%%%%%%%%%%%%%%%%%%%%%%%%%%%%%%%%%%%%%%%%%%%%%%%%%
\subsection{Non-central Chi Squared distribution}

\subsubsection*{PDF}
\[
    P(x) = {\frac {1}{2}}e^{-(x+\lambda )/2}\left({\frac {x}{\lambda }}\right)
    ^{k/4-1/2}I_{k/2-1}({\sqrt {\lambda x}})
\]
where $k$ is the degrees of freedom; $\lambda$ is the distance parameter;

\subsubsection*{CDF}
\[
    D(x) = 1-Q_{\frac {k}{2}}\left({\sqrt {\lambda }},{\sqrt {x}}\right)
\]
where $Q_M(a,b)$ is Marcum Q-function.

\subsubsection*{Mean}
\[
    k + \lambda
\]

\subsubsection*{Variance}
\[
    2(k+2\lambda)
\]

\subsubsection*{Skewness}
\[
    \frac{2^{3/2}(k+3\lambda)}{(k+2\lambda)^{3/2}}
\]

Reference:
\href{https://en.wikipedia.org/wiki/Noncentral_chi-squared_distribution}{Wikipedia}

%%%%%%%%%%%%%%%%%%%%%%%%%%%%%%%%%%%%%%%%%%%%%%%%%%%%%%%%%%%%%%%%%%
\subsection{Normal distribution}

\subsubsection*{PDF}
\[
    P(x) = {\frac{1}{\sigma{\sqrt {2\pi }}}}
        \exp{(-{\frac {1}{2}} 
        \left({\frac {x-\mu }{\sigma }}\right)^{2})}
\]
where $\mu$ is the expection of the distribution; 
$\sigma$ is the standard deviation.

\subsubsection*{CDF}
\[
    D(x) = {\frac{1}{2}} \left[1+\operatorname {erf} 
        \left({\frac {x-\mu }{\sigma {\sqrt {2}}}}\right)\right]
\]
where $erf(a)$ is the error function.

\subsubsection*{Mean}
\[
    \mu
\]

\subsubsection*{Variance}
\[
    \sigma^2
\]

\subsubsection*{Skewness}
\[
    0
\]

Reference:
\href{https://en.wikipedia.org/wiki/Normal_distribution}{Wikipedia}

%%%%%%%%%%%%%%%%%%%%%%%%%%%%%%%%%%%%%%%%%%%%%%%%%%%%%%%%%%%%%%%%%%
\subsection{Rayleigh distribution}

\subsubsection*{PDF}
\[
    P(x) = {\frac {x}{\sigma ^{2}}}e^{-x^{2}/\left(2\sigma ^{2}\right)}
\]

\subsubsection*{CDF}
\[
    D(x) = 1 - e^{-x^{2}/\left(2\sigma ^{2}\right)}
\]

\subsubsection*{Mean}
\[
    \sigma \sqrt{\frac{\pi}{2}}
\]

\subsubsection*{Variance}
\[
    {\frac {4-\pi }{2}}\sigma ^{2}
\]

\subsubsection*{Skewness}
\[
    {\frac {2{\sqrt {\pi }}(\pi -3)}{(4-\pi )^{3/2}}}
\]

Reference:
\href{https://en.wikipedia.org/wiki/Rayleigh_distribution}{Wikipedia}

%%%%%%%%%%%%%%%%%%%%%%%%%%%%%%%%%%%%%%%%%%%%%%%%%%%%%%%%%%%%%%%%%%
\subsection{Standard Normal distribution}

\subsubsection*{PDF}
\[
    P(x) = \frac{1}{\sqrt {2\pi }}
        \exp{(-{\frac {1}{2}} x^{2})}
\]
where $\mu$ is the expection of the distribution; 
$\sigma$ is the standard deviation.

\subsubsection*{CDF}
\[
    D(x) = {\frac{1}{2}} \left[1+\operatorname {erf} 
        \left( {\frac{x}{{\sqrt {2}}}} \right) \right]
\]
where $\operatorname {erf}(a)$ is the error function.

\subsubsection*{Mean}
\[
    0
\]

\subsubsection*{Variance}
\[
    1
\]

\subsubsection*{Skewness}
\[
    0
\]

%%%%%%%%%%%%%%%%%%%%%%%%%%%%%%%%%%%%%%%%%%%%%%%%%%%%%%%%%%%%%%%%%%
\subsection{Uniform distribution (Continuous)}

\subsubsection*{PDF}
\[
    P(x) = \begin{cases}
        {\frac {1}{b-a}}&{\text{for }}x\in [a,b]\\
        0&{\text{otherwise}}
    \end{cases}
\]
where $a$ is the lower bound; $b$ is the upper bound.

\subsubsection*{CDF}
\[
    D(x) = \begin{cases}
        0&{\text{for }}x<a\\
        {\frac {x-a}{b-a}}&{\text{for }}x\in [a,b]\\
        1&{\text{for }}x>b
    \end{cases}
\]

\subsubsection*{Mean}
\[
    \frac{1}{2}(a+b)
\]

\subsubsection*{Variance}
\[
    \frac{1}{12}(b-a)^2
\]

\subsubsection*{Skewness}
\[
    0
\]

Reference:
\href{https://en.wikipedia.org/wiki/Continuous_uniform_distribution}{Wikipedia}

%%%%%%%%%%%%%%%%%%%%%%%%%%%%%%%%%%%%%%%%%%%%%%%%%%%%%%%%%%%%%%%%%%
%%%%%%%%%%%%%%%%%%%%%%%%%%%%%%%%%%%%%%%%%%%%%%%%%%%%%%%%%%%%%%%%%%

\section{Mixture distribution}

A \textbf{mixture distribution} is the probability distribution of a random variable 
that is derived from a collection of other random varibales. 
The probability density function (and the cumulative distribution function) 
can be expressed as the a convex combination of other distribution functions.
The individual distributions that are combined to form the mixture distribution are called 
the \textbf{mixture components}. The weights associated with each component are called the \textbf{mixture weights}.
(\href{https://www.overleaf.com/learn/latex/Bold,_italics_and_underlining}{Wikipedia})


\subsubsection*{PDF}
Given the mixture components' PDF ${P_1(x), ..., P_n(x)}$ and weights ${w_1,...,w_n}$,
the mixture's PDF is a convex combination:
\[
    P_\text{mixture}(x) = \sum^n_{i=1} w_i P_i(x)
\]

\subsubsection*{CDF}
Given the mixture components' CDF ${D_1(x), ..., D_n(x)}$ and weights ${w_1,...,w_n}$,
the mixture's CDF is a convex combination:
\[
    D_\text{mixture}(x) = \sum^n_{i=1} w_i D_i(x)
\]


The moments of a mixture distribution are not as mathematically simple as PDF or CDF. 
To acquire the mathematically expressions for Mean, Variance, and Skewness, we need to 
clarify the three types of moments:
\begin{enumerate}
    \item $k^\text{th}$ non-central moment: $\mu^{(k)} = \mathbb{E}[x^k]$
    \item $k^\text{th}$ central moment: $\mu^{(k)}_c = \mathbb{E}[(x-\mu^{(1)})^k]$
    \item $k^\text{th}$ standardized moment: $\mu^{(k)}_s = \mathbb{E}[(\frac{x-\mu^{(1)}}{\sigma})^k]$
\end{enumerate}
, where $^{(k)}$ denotes the $k^\text{th}$ moment.



\subsubsection*{Mean}
Mean is also known as the first non-central moment. 
The $k^\text{th}$ non-central moment of a random variable can be rewritten in terms of integrals as:
\begin{align}
    \mu^{(k)} &= \mathbb{E}[x^k] \\
              &= \int_{-\infty}^{\infty} x^k f(x) dx \\
              &= \int_{-\infty}^{\infty} x^k \sum^n_{i=1} w_i f_i(x) dx \\
              &= \sum^n_{i=1} w_i \int_{-\infty}^{\infty} x^k f_i(x) dx \\
              &= \sum^n_{i=1} w_i \mathbb{E}_{f_i}[x^k] \\
              &= \sum^n_{i=1} w_i \mu_i^{(k)}
\end{align}
, where $\mu_i^{(k)}$ is the $k^\text{th}$ non-central moment of distribution function $f_i$.

Given the mixture components' mean ${\mu_1, ..., \mu_n}$ and weights ${w_1,...,w_n}$,
it is easy to see the mixture's mean is just a convex combination:
\[
    \mu^{(1)}_\text{mixture} = \sum^n_{i=1} w_i \mu^{(1)}_i
\]



\subsubsection*{Variance}
Variance is also known as the second central moment. The $k^\text{th}$ central moment 
follows a deriavation similar to the non-central moment:
\begin{align}
    \sigma^2 \coloneqq \mu^{(2)}_c &= \mathbb{E}[(x-\mu^{(1)})^{2}] \\
        &= \mathbb{E}[x^2] - (\mathbb{E}[x])^2 \\
        &= \mu^{(2)} - (\mu^{(1)})^2 \\
        &= \sum^n_{i=1} w_i \mu^{(2)}_i - \left(\sum^n_{i=1} w_i \mu^{(1)}_i \right)^2  \\
        &= \sum^n_{i=1} w_i \left( \sigma^2_i + (\mu^{(1)}_i)^2 \right) - \left(\sum^n_{i=1} w_i \mu^{(1)}_i \right)^2  \\
        &= \sum^n_{i=1} w_i \sigma^2_i + \sum^n_{i=1} w_i (\mu^{(1)}_i)^2 - \left(\sum^n_{i=1} w_i \mu^{(1)}_i \right)^2 
\end{align}
, where $\sigma_i^2$ is $i^{th}$ component's variance, and $\mu^{(1)}_i$ is $i^{th}$ component's mean.

With simplified the notations, we get a familiar expression for the mixture's variance.
\[
    \sigma^2_\text{mixture} = \sum^n_{i=1} w_i (\sigma^2_i+\mu^2_i) - \mu^2_\text{mixture}
\]



\subsubsection*{Skewness}
Skewness is also known as the third standardized moment. 
\begin{align}
    \mu^{(3)}_s &= \mathbb{E}[ \left( \frac{x-\mu^{(1)}}{\sigma} \right) ^3] \\
                &= \frac{\mathbb{E}[x^3] - 3 \mu^{(1)} \mathbb{E}[x^2] + 3 (\mu^{(1)})^2 \mathbb{E}[x] - (\mu^{(1)})^3}{\sigma^3} \\
                &= \frac{\mathbb{E}[x^3] - 3 \mu^{(1)} \sigma^2 - (\mu^{(1)})^3}{\sigma^3} 
\end{align}

To rewrite it in terms of the skewness of each component $(\mu^{(3)}_s)_i$, let's start by rewriting the third non-central moment:
\[
    \mathbb{E}[x^3] = \sum^n_{i=1} w_i \mu^{(3)}_i = \sum^n_{i=1} w_i \left( \sigma^3_i (\mu^{(3)}_s)_i + 3 \mu^{(1)}_i \sigma^2_i + (\mu^{(1)}_i)^3 \right)
\]

Now the Skewness of the mixture can be rewritten to:
\[
    \mu^{(3)}_s = \frac{ \sum^n_{i=1} w_i \left( \sigma^3_i (\mu^{(3)}_s)_i + 3 \mu^{(1)}_i \sigma^2_i + (\mu^{(1)}_i)^3 \right)  - 3 \mu^{(1)} \sigma^2 - (\mu^{(1)})^3 }{\sigma^3}
\]
where $\mu^{(1)}$ is the mean of the mixture, and $\sigma^2$ is the variance of the mixture.