\section{Usage in C++}

A good place to see example usages is the gtest files in the \textbf{tests} directory.

Because all distribution classes are derived \(\code{probDistr}\) class in \(\code{probDistr.h}\), 
their interfaces are the same. That is, you can call the following methods 
in all probability distribution classes:
\begin{enumerate}
    \item pdf
    \item cdf
    \item mean
    \item stddev
    \item variance
    \item skewness
    \item hash
    \item print
    \item isEqual\_tol
    \item isEqual\_ulp
    \item getID
\end{enumerate}

In addition to the interface methods above, the Mixture distribution has some generic container methods.
Please see \(\code{tests/unit_test/tst_disMixture.cpp}\) for reference.


%%%%%%%%%%%%%%%%%%%%%%%%%%%%%%%%%%%%%%%%%%%%%%%%%%%%%%%%%%

\subsection{How to add a new probability distribution?}

Since probability distribution classes inherit from \(\code{probDistr}\) class in \(\code{probDistr.h}\), 
a new probability classes should implement:
\begin{enumerate}
    \item the pure virtual functions in \(\code{probDistr}\).
    \item a hasher.
    \item a cloner which duplicates an object that is raw or is managed by smart-pointer.
    \item a serializer for human-friendly text to STDOUT.
    \item two comparers with different type of thresholds -- floating-point values and ULPs.
    \item a unique ID that identifies the distribution.
\end{enumerate} 


\begin{minted}{cpp}
class probDistr {
public:

    virtual double pdf(const double=0) const    = 0;
    virtual double cdf(const double=0) const    = 0;
    virtual double mean() const                 = 0;
    virtual double stddev() const               = 0;
    virtual double variance() const             = 0;
    virtual double skewness() const             = 0;
    
    virtual std::size_t hash() const noexcept {
        std::size_t seed = 0;
        combine_hash(seed, id);
        return seed;
    }

    virtual std::unique_ptr<probDistr> cloneUnique() const = 0;
    virtual probDistr* clone() const = 0;    // Return type can be Covariant.

    friend std::ostream& operator << (std::ostream&, const probDistr&);
    virtual void print(std::ostream&) const = 0;

    virtual bool isEqual_tol(const probDistr&, const double) const = 0;
    virtual bool isEqual_ulp(const probDistr&, const unsigned) const = 0;

    virtual dFuncID getID() const {return id;};
    const dFuncID id = dFuncID::BASE_DISTR;

    ...
}
\end{minted}


In addition, it is recommanded to implement those to align the feature set with existing distributions:
\begin{enumerate}
    \item getters for distribution parameters.
    \item re-direct std::hash to call custome hasher.
\end{enumerate}


For example, below is my implementation of Rician distribution.

\begin{minted}{cpp}
class disRician : public probDistr {
private:

    double nu;      // distance
    double sigma;   // scale

public:
    template<class T, class U>
    requires std::is_arithmetic_v<T> && std::is_arithmetic_v<U> 
    disRician(const T distance, const U scale) {
        nu = std::abs(distance);
        sigma = scale;
    }
    disRician() = delete;
    ~disRician() = default;

    double pdf(const double x) const override {
        const double s2inv = 1/(sigma*sigma);
        const double x_e = exp(-(x*x+nu*nu)*0.5*s2inv) * std::cyl_bessel_i(0,x*nu*s2inv);
        return x * s2inv * x_e;
    }

    double cdf(const double x) const override {
        const double s_inv = 1/sigma;
        return 1 - marcumQ(1, nu*s_inv, x*s_inv);
    }

    double mean() const override {
        const double x = -0.5*nu*nu/(sigma*sigma);
        const double lague = exp(x/2) * 
            ((1-x)*std::cyl_bessel_i(0,-0.5*x) - x*std::cyl_bessel_i(1,-0.5*x));
        return sigma * SQRT_PI_2 * lague;
    }

    double stddev() const override {
        return std::sqrt(variance());
    }

    double variance() const override {
        const double x = -0.5*nu*nu/(sigma*sigma);
        const double lague = exp(x/2) * 
            ((1-x)*std::cyl_bessel_i(0,-0.5*x) - x*std::cyl_bessel_i(1,-0.5*x));
        return 2*sigma*sigma + nu*nu - M_PI_2*sigma*sigma*lague*lague;
    }

    double skewness() const override {
        throw std::runtime_error("Rician distribution's skewness is too complicated.");
        return 0;
    }

    inline std::size_t hash() const noexcept {
        std::size_t seed = 0;
        combine_hash(seed, char(id));
        combine_hash(seed, nu);
        combine_hash(seed, sigma);
        return seed;
    } 

    std::unique_ptr<probDistr> cloneUnique() const override {
        return std::make_unique<disRician>(static_cast<disRician const&>(*this));
    };

    disRician* clone() const override {
        return new disRician(*this);
    }

    void print(std::ostream& output) const override {
        output << "Rician distribution -- nu = " << nu << " sigma = " << sigma;
    }

    bool isEqual_tol(const probDistr& o, const double tol) const override {
        const disRician& oo = dynamic_cast<const disRician&>(o);
        bool r = true;
        r &= isEqual_fl_tol(nu, oo.nu, tol);
        r &= isEqual_fl_tol(sigma, oo.sigma, tol);
        return r;
    }

    bool isEqual_ulp(const probDistr& o, const unsigned ulp) const override {
        const disRician& oo = dynamic_cast<const disRician&>(o);
        bool r = true;
        r &= isEqual_fl_ulp(nu, oo.nu, ulp);
        r &= isEqual_fl_ulp(sigma, oo.sigma, ulp);
        return r;
    }

    auto p_distance() const noexcept {
        return nu;
    }

    auto p_scale() const noexcept {
        return sigma;
    }
};

template<>
class std::hash<statanaly::disRician> {
public:
    std::size_t operator() (const statanaly::disRician& d) const {
        return d.hash();
    }
};
\end{minted}